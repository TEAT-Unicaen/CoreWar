\documentclass{beamer}
\usepackage[utf8]{inputenc}
\usepackage[french]{babel}
\usepackage[T1]{fontenc}
\usepackage{graphicx}

% Theme
\usetheme{metropolis}

% Title page
\title{Projet : CoreWar}
\author{Etienne BOSSU\and{}Amand HENRY\and{}Tom ROUSEE\and{}Theo SICOT}
\date{\today}

% Slides
\begin{document}

\begin{frame}
    \titlepage
\end{frame}

\begin{frame}{Sommaire}
    \tableofcontents
\end{frame}

\section{Introduction}
\begin{frame}{Introduction}
    \begin{itemize}
        \item But du projet : créer un jeu de CoreWar et un algorithme génétique permettant de créer des programmes efficaces avec Java.
        \vspace{\baselineskip}
        \item CoreWar : jeu de programmation où deux programmes s'affrontent dans une machine virtuelle (MARS).
    \end{itemize}
\end{frame}

\begin{frame}{Identification des objectifs :}
    Pour réaliser ce projet nous avons identifier plusieurs points clefs  du sujet:
    \vspace{\baselineskip}
    \begin{itemize}
        \item RedCode
        \item Machine MARS
        \item Algorithme génétique
        \item Interface graphique
    \end{itemize}
\end{frame}

\section{RedCode}
\begin{frame}
    \frametitle{RedCode}
    Le RedCode :
    \begin{itemize}
        \item Langage de programmation type Assembleur
        \item Utilisé pour écrire les programmes des guerriers
        \item 11 instructions / 4 modes d'adressage
    \end{itemize}
\end{frame}

\section{Machine MARS}
\begin{frame}
    \frametitle{MARS : Memory Array Redcode Simulator}
    C'est dans cette machine que les guerriers s'affrontent. Elle gère le déroulement de la partie, interprète le RedCode et gère la mémoire.
\end{frame}


\begin{frame}{Mémoire}
    \begin{figure}
        \centering
        \includegraphics[width = 8cm]{img/umlMemoire.png}
        \caption{Classes de la mémoire}
    \end{figure}
\end{frame}

\begin{frame}{Le Déroulement d'une partie}
    \begin{itemize}
        \item Appels dans la Process Queue (SPL et DAT)
        \item Exécution des instructions
    \end{itemize}
\end{frame}

\begin{frame}{Debug de la Machine}
    \begin{figure}
        \centering
        \includegraphics[width = 0.45\textwidth]{img/debugMode.png}
        \hfill
        \includegraphics[width = 0.45\textwidth]{img/loopProtection.png}
        \caption{Le mode Debug et L'erreur de la boucle infinie}
    \end{figure}
\end{frame}

\section{Interface graphique}
\begin{frame}
    \frametitle{Interface graphique}
    Elle permet de visualiser le déroulement d'\textbf{une seule partie} de CoreWar.
    \begin{figure}
        \centering
        \includegraphics[width = 8cm]{img/display.jpg}
        \caption{L'interface graphique}
    \end{figure}
\vspace{-5.5pt}
\end{frame}

\section{Algorithme génétique}
\begin{frame}{Algorithme génétique}
    Le but est de créer des programmes de plus en plus efficaces.
    \begin{figure}
        \centering
        \includegraphics[width = 8cm]{img/fonctionnement.png}
        \caption{Fonctionnement de l'algorithme génétique}
    \end{figure}
\end{frame}

\begin{frame}{Fonctionnement de la reproduction}
    \begin{figure}
        \centering
        \includegraphics[width = 8cm]{img/crossover.png}
        \caption{Crossover}
    \end{figure}
\end{frame}

\begin{frame}{Algorithme génétique : Entrainement}
    \begin{itemize}
        \item Le CheatCode
        \item Entrainement à partir de la population
        \item La méthode du Challenger
    \end{itemize}
\end{frame}

\section{Résultats et Conclusion}
\begin{frame}
    \frametitle{Résultats}
    \begin{figure}
        \centering
        \includegraphics[width = 8cm]{img/graphique3.png}
        \caption{Graphique résultant de l'entrainement}
    \end{figure}
\end{frame}

\begin{frame}{}
    \begin{center}
        \Huge Conclusion
    \end{center}
\end{frame}

\end{document}