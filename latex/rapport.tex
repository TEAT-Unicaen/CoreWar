\documentclass[a4paper, 10pt]{article}

%Internal Packages
\usepackage[utf8]{inputenc}
\usepackage[french]{babel}
\usepackage[T1]{fontenc}
\usepackage{graphicx}
\usepackage{hyperref}       % Pour les liens hypertextes

%Variables
\title{Rapport de CoreWar}
\author{Amand Henry\and{}Théo Sicot\and{}Etienne Bossu\and{}Tom Rousée}
\date{\today{}}

%Document
\begin{document}
    \maketitle{}
    \newpage{}
    \tableofcontents{}
    \newpage{}

    % Introduction
    \begin{section}{Introduction} \label{sec:introduction}
        \par
            Le CoreWar est un jeu de programmation où deux programmes sont opposés pour prendre le contrôle d'une machine virtuelle appelée MARS (Memory Array Redcode Simulator). Ces programmes sont écrits dans un langage proche de l'assembleur appelé RedCode qui a ses commandes et sa syntaxe spécifique. Ces programmes, appelés "Guerriers" ont pour objectif d'être les derniers à s'exécuter en faisant se terminer toutes les instances du programme adverse.
            \medskip
        \par
            Au début les joueurs concevaient leurs programmes par eux mêmes mais, depuis l'apparition des algorithmes génétiques, les joueurs n'essaient plus de créer leurs guerriers. L'ordinateur optimise les meilleurs programmes possible pour gagner la partie.
            \bigskip
        \par
            Dans ce projet nous avons créé tout ce qu'il y a besoin pour faire une partie de CoreWar : la machine MARS, l'interpréteur pour le RedCode ainsi qu'un algorithme génétique qui permet de créer des guerriers et de quoi afficher la partie. 
    \end{section}

    % Méthodologie
    \begin{section}{Les Parties du projet} \label{sec:methodologie}
        Décrivez ici comment vous avez abordé le projet, les outils et langages de programmation utilisés, ainsi que les différentes étapes de votre développement.
    \end{section}

    % Résultats
    \begin{section}{Résultats}\label{sec:resultats}
        Présentez les résultats obtenus, avec éventuellement des captures d'écran du jeu en action, des graphiques de performances, etc.
    \end{section}

    % Discussion
    \begin{section}{Discussion}\label{sec:discussion}
        Discutez de l'impact de vos choix de conception, des difficultés rencontrées, et de la manière dont elles ont été surmontées.
    \end{section}

    % Conclusion
    \begin{section}{Conclusion}\label{sec:conclusion}
        Résumez les principaux résultats, ce que vous avez appris durant le projet, et les perspectives futures.
    \end{section}

    % Références
    \bibliographystyle{plain}
    \bibliography{references}
    
\end{document}