\documentclass[a4paper, 10pt]{article}

%Internal Packages
\usepackage[utf8]{inputenc}
\usepackage[french]{babel}
\usepackage[T1]{fontenc}
\usepackage{graphicx}
\usepackage{hyperref}       % Pour les liens hypertextes

%Variables
\title{Rapport de CoreWar}
\author{Amand Henry\and{}Théo Sicot\and{}Etienne Bossy\and{}Tom Rousée}
\date{\today{}}

%Document
\begin{document}
    \maketitle{}
    \newpage{}
    \tableofcontents{}
    \newpage{}

    % Introduction
    \begin{section}{Introduction}\label{sec:introduction}
        Dans cette partie, nous abordons le thème central de notre rapport : le CoreWar. Nous détaillons ici l'essence de ce jeu de programmation, retraçons brièvement son historique depuis sa création jusqu'à son évolution actuelle, et exposons les finalités spécifiques de notre projet en lien avec ce contexte.
    \end{section}

    % Méthodologie
    \begin{section}{Méthodologie}\label{sec:methodologie}
        Décrivez ici comment vous avez abordé le projet, les outils et langages de programmation utilisés, ainsi que les différentes étapes de votre développement.
    \end{section}

    % Résultats
    \begin{section}{Résultats}\label{sec:resultats}
        Présentez les résultats obtenus, avec éventuellement des captures d'écran du jeu en action, des graphiques de performances, etc.
    \end{section}

    % Discussion
    \begin{section}{Discussion}\label{sec:discussion}
        Discutez de l'impact de vos choix de conception, des difficultés rencontrées, et de la manière dont elles ont été surmontées.
    \end{section}

    % Conclusion
    \begin{section}{Conclusion}\label{sec:conclusion}
        Résumez les principaux résultats, ce que vous avez appris durant le projet, et les perspectives futures.
    \end{section}

    % Références
    \bibliographystyle{plain}
    \bibliography{references}
    
\end{document}